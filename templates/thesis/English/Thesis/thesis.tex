%% SE-KCSS-Style v1.3
%% SE-KCSS-Style EN v1.3
\documentclass[10pt]{book}
\usepackage{fixltx2e}
\usepackage[resetfonts]{cmap}
\usepackage{nameref}
\usepackage[%
  language=english,paper=a4paper,largepaper=true,%
  algorithm=false,%
  biblatexstyle=authoryear-square,biblatexOptions={natbib=true,backend=bibtex},%
  acronymOptions={smaller,printonlyused},%,withpage
]{ifiseries}
\usepackage[format=hang]{caption}[2008/08/24]
\usepackage{textcomp}
\usepackage{todonotes}
\newcommand{\thesistitlepage}[5]{\gentitlepage{#1}{#2}{#3\\\Large\vspace{5ex}#4}{\Large\textsc{Kiel University\\Department of Computer Science \\Software Engineering Group}\\\vspace{10ex}\begin{tabular}{rl}Advised by: & Prof. Dr. Wilhelm Hasselbring \\ & #5 \\\end{tabular}}}
\ExecuteBibliographyOptions{sortcase=false,babel=other,backref=true,abbreviate=false}
\ExecuteBibliographyOptions{isbn=false,url=true,doi=false,eprint=false}
\addbibresource{bibliography.bib}

\hypersetup{bookmarksdepth=3}
\hypersetup{bookmarksopen=true}
\hypersetup{bookmarksopenlevel=0}
\hypersetup{bookmarksnumbered=true}

\newcommand{\TODO}[1]{\todo[inline]{#1}}

\endinput


\hypersetup{pdftitle=My Fake Bachelor's Thesis with a Long Title over Three Lines}
\hypersetup{pdfauthor=John Q. Pregraduate}
\hypersetup{pdfsubject=Bachelor's Thesis }
\hypersetup{pdfkeywords=}

\begin{document}
\frontmatter
  \thesistitlepage
    {My Fake Bachelor's Thesis \\[.1em]With a Long Title \\[.1em]Over Three Lines}% Title
    {Bachelor's Thesis}% Thesistype
    {John Q. Pregraduate}% Name
    {\today}% Date
    {additional advisor} % additional advisors (add more than one with: name1 \\& name2 \\& name3

  \eidesstatt{}

  \chapter*{Abstract}
    \blindtext

  \tableofcontents{}
  %\listoffigures{}\listoftables{}\lstlistoflistings{}
  %\chapter*{List of Acronyms}
  %% List of acronyms

\mainmatter

%% the actual thesis
%% use \include{filename} and \includeonly{filename} to organize your thesis

\chapter{Introduction}
  This is my introduction.
  \citet{Shaw2003} wrote a paper with hints on how to write good software engineering research papers.
  By the way, this was an example for using the \textit{natbib} command \texttt{\textbackslash{}citet\{\}}.
  
  \textit{AspectJ} is tool to weave cross-cutting concerns into Java programs~\citep{AspectJ}. By the way, this was an example for using the \textit{natbib} command \texttt{\textbackslash{}citep\{\}}.
  
  Kieker: \citep{Rohr2008, Hoorn2009, Hoorn2012}
  
  Article Example: \citep{Frey2011}
  
  We will now demonstrate how to use figures (see \autoref{fig:figure}).
  
  \begin{figure}[t]%
    \centering%
    \includegraphics[width=0.3\textwidth]{img/template_circle.pdf}%
    \caption{A circle}%
    \label{fig:figure}%
  \end{figure}%

  You should use diagrams, graphics, tables etc. to explain your topics.
  Further hints on writing a thesis can be found at \url{http://se.informatik.uni-kiel.de/research/scientific-work/} and \url{http://www.twenzel.de/}.

  \section{Motivation}
    \blindtext
  \section{Goals}
    \blindtext
  \section{Document Structure}
    \autoref{chp:Foundations} presents everything one must know. The conclusions follow in \autoref{chp:Conclusions}. \TODO{Adjust this}

\chapter{Foundations and Technologies}\label{chp:Foundations}
  \section{Research Topic or Technology 1}
    \blindtext
  \section{Research Topic or Technology 2}
    \blindtext
  \section{Research Topic or Technology m}
    \blindtext

\chapter{Approach Part 1}\label{chp:Approach1}
  \section{Approach Part 1 Subsection 1}
    \blindtext
  \section{Approach Part 1 Subsection 2}
    \blindtext
  \section{Approach Part 1 Subsection n}
    \blindtext

\chapter{Approach Part 2}\label{chp:Approach2}
  \section{Approach Part 2 Subsection 1}
    \blindtext
  \section{Approach Part 2 Subsection 2}
    \blindtext
  \section{Approach Part 2 Subsection o}
    \blindtext

\chapter{Evaluation}\label{chp:Evaluation}
  \blindtext

\chapter{Related Work}\label{chp:Related}
  \blindtext

\chapter{Conclusions and Future Work}\label{chp:Conclusions}
  \section{Conclusions}
    \blindtext
  \section{Future Work}
    \blindtext

%%

\backmatter
  \tocbibliography

\end{document}
