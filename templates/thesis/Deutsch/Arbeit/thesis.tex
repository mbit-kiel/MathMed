%% SE-KCSS-Style v1.3
%% SE-KCSS-Style DE v1.3
\documentclass[10pt]{book}
\usepackage{fixltx2e}
\usepackage[resetfonts]{cmap}
\usepackage{nameref}
\usepackage[%
  language=german,paper=a4paper,largepaper=true,%
  algorithm=false,%
  biblatexstyle=authoryear-square,biblatexOptions={natbib=true,backend=bibtex},%
  acronymOptions={smaller,printonlyused},%,withpage
]{ifiseries}
\usepackage[format=hang]{caption}[2008/08/24]
\usepackage{textcomp}
\usepackage{todonotes}
\newcommand{\thesistitlepage}[5]{\gentitlepage{#1}{#2}{#3\\\Large\vspace{5ex}#4}{\Large\textsc{Christian-Albrechts-Universit\"{a}t zu Kiel\\Institut f\"{u}r Informatik\\ Arbeitsgruppe Software Engineering }\\\vspace{10ex}\begin{tabular}{rl}Betreut durch: & Prof. Dr. Wilhelm Hasselbring \\ & #5 \\\end{tabular}}}
\ExecuteBibliographyOptions{sortcase=false,babel=other,backref=true,abbreviate=false}
\ExecuteBibliographyOptions{isbn=false,url=true,doi=false,eprint=false}
\addbibresource{bibliography.bib}

\hypersetup{bookmarksdepth=3}
\hypersetup{bookmarksopen=true}
\hypersetup{bookmarksopenlevel=0}
\hypersetup{bookmarksnumbered=true}

\newcommand{\TODO}[1]{\todo[inline]{#1}}

\endinput


\hypersetup{pdftitle=Meine Fake Bachelorarbeit mit einem sehr sehr sehr lange Titel}
\hypersetup{pdfauthor=John Q. Pregraduate}
\hypersetup{pdfsubject=Bachelorarbeit}
\hypersetup{pdfkeywords=}

\begin{document}
\frontmatter
  \thesistitlepage
    {Meine Fake Bachelorarbeit \\[.1em]mit einem sehr sehr \\[.1em]sehr lange Titel}% Title
    {Bachelorarbeit}% Thesistype
    {John Q. Pregraduate}% Name
    {\today}% Date
    {weiterer Betreuer} % additional advisors (add more than one with: name1 \\& name2 \\& name3

  \eidesstatt{}

  \chapter*{Zusammenfassung}
    \blindtext

  \tableofcontents{}
  %\listoffigures{}\listoftables{}\lstlistoflistings{}
  %\chapter*{List of Acronyms}
  %% List of acronyms

\mainmatter

%% the actual thesis
%% use \include{filename} and \includeonly{filename} to organize your thesis

\chapter{Einleitung}
  Dies ist die Einleitung
  \citet{Shaw2003} haben ein ganz tolles Papier geschrieben
  Dies war ein Beispiel für den \textit{natbib} Befehl für \texttt{\textbackslash{}citet\{\}}.
  
  \textit{AspectJ} ist ein tolles Tool~\citep{AspectJ}. Dies war ein Beispiel für den \textit{natbib} Befehl \texttt{\textbackslash{}citep\{\}}.
  
  Kieker: \citep{Rohr2008, Hoorn2009, Hoorn2012}
  
  Artikel Beispiel: \citep{Frey2011}
  
  Wir werden nun die Verwendung von Bildern veranschaulichen (siehe \autoref{fig:figure}).
  
  \begin{figure}[t]%
    \centering%
    \includegraphics[width=0.3\textwidth]{img/template_circle.pdf}%
    \caption{Ein Kreis}%
    \label{fig:figure}%
  \end{figure}%

  Weitere Hinweise auf \url{http://se.informatik.uni-kiel.de/research/scientific-work/} und \url{http://www.twenzel.de/}.

  \section{Motivation}
    \blindtext
  \section{Ziele}
    \blindtext
  \section{Aufbau}
    \autoref{chp:Foundations} präsentiert alle schönen Grundlagen. Das Fazit ist in  \autoref{chp:Conclusions}. \TODO{Anpassen!}

\chapter{Grundlagen und Technologien}\label{chp:Foundations}
  \section{Grundlage oder Technologie 1}
    \blindtext
  \section{Grundlage oder Technologie 2}
    \blindtext
  \section{Grundlage oder Technologie n}
    \blindtext

\chapter{Ansatz Teil 1}\label{chp:Approach1}
  \section{Ansatz Teil 1 Unterkapitel 1}
    \blindtext
  \section{Ansatz Teil 1 Unterkapitel 2}
    \blindtext
  \section{Ansatz Teil 1 Unterkapitel n}
    \blindtext

\chapter{Ansatz Teil 2}\label{chp:Approach2}
  \section{Ansatz Teil 2 Unterkapitel 1}
    \blindtext
  \section{Ansatz Teil 2 Unterkapitel 2}
    \blindtext
  \section{Ansatz Teil 2 Unterkapitel n}
    \blindtext

\chapter{Evaluierung}\label{chp:Evaluation}
  \blindtext

\chapter{Verwandte Arbeiten}\label{chp:Related}
  \blindtext

\chapter{Fazit und Ausblick}\label{chp:Conclusions}
  \section{Fazit}
    \blindtext
  \section{Ausblick}
    \blindtext

%%

\backmatter
  \tocbibliography

\end{document}
