\documentclass{article}

%%%%%%%%%%%%%%%%%%%%%%%%%%%%%%%%%%%%%%%%%%%%%%%%%%%%%%%%%%%%%%%%%%%%%%%%%%%%%%%%
% German language settings. Allows the use of umlauts.                         %
%%%%%%%%%%%%%%%%%%%%%%%%%%%%%%%%%%%%%%%%%%%%%%%%%%%%%%%%%%%%%%%%%%%%%%%%%%%%%%%%
\usepackage[utf8]{inputenc}
\usepackage[T1]{fontenc}
\usepackage[ngerman]{babel}

%%%%%%%%%%%%%%%%%%%%%%%%%%%%%%%%%%%%%%%%%%%%%%%%%%%%%%%%%%%%%%%%%%%%%%%%%%%%%%%%
% These packages provide fonts for the document, but may require manual        %
% installation. The usual procedure (on Linux) is the following:               %
% 1. Download the package files                                                %
% 2. Extract the files to ~/texmf/tex/latex/newpackage                         %
% 3. If never done before, run "texhash ~/texmf" in a terminal                 %
% 4. run "sudo texhash" in a terminal                                          %
%                                                                              %
% At least one of these fonts is most likely available on your system. If this %
% is not the case, you can simply comment all package imports to use the       %
% default font "Computer Modern Roman".                                        %
%%%%%%%%%%%%%%%%%%%%%%%%%%%%%%%%%%%%%%%%%%%%%%%%%%%%%%%%%%%%%%%%%%%%%%%%%%%%%%%%

% TeX Gyre Times 
\usepackage{tgtermes}

% Fourier Font
% \usepackage{fourier}

% A somewhat outdated Times alternative
% \usepackage{mathptmx}

% A Palatino substitute with adjusted line spreading.
% \usepackage{mathpazo}
% \linespread{1.05}

%%%%%%%%%%%%%%%%%%%%%%%%%%%%%%%%%%%%%%%%%%%%%%%%%%%%%%%%%%%%%%%%%%%%%%%%%%%%%%%%
% This package provides the useful commands \set and \Set, that can be used to %
% denote sets. The \set command uses rigid brackets, whereas the \Set command  %
% employs flexible brackets to enclose the set.                                %
% Examples:                                                                    %
%  A \setminus \set{a_0}                                                       %
%  \set{x \in A | x \le 4}                                                     %
%  \Set{\int_{\Omega} f d\mu | f \in \mathcal{L}_2(\Omega, \mathfrak{A}, \mu)} %
%%%%%%%%%%%%%%%%%%%%%%%%%%%%%%%%%%%%%%%%%%%%%%%%%%%%%%%%%%%%%%%%%%%%%%%%%%%%%%%%

\usepackage{braket}

%%%%%%%%%%%%%%%%%%%%%%%%%%%%%%%%%%%%%%%%%%%%%%%%%%%%%%%%%%%%%%%%%%%%%%%%%%%%%%%%
% Several packages for mathematical symbols.                                   %
%%%%%%%%%%%%%%%%%%%%%%%%%%%%%%%%%%%%%%%%%%%%%%%%%%%%%%%%%%%%%%%%%%%%%%%%%%%%%%%%

\usepackage{amsmath, amsfonts, amssymb}

%%%%%%%%%%%%%%%%%%%%%%%%%%%%%%%%%%%%%%%%%%%%%%%%%%%%%%%%%%%%%%%%%%%%%%%%%%%%%%%%
% Provides double-stroked black board letters. In general \mathbb{LETTER is    %
% preferrable, but this package also provides a double-stroked 1: mathds{1}.   %
%%%%%%%%%%%%%%%%%%%%%%%%%%%%%%%%%%%%%%%%%%%%%%%%%%%%%%%%%%%%%%%%%%%%%%%%%%%%%%%%

% \usepackage{dsfont}

%%%%%%%%%%%%%%%%%%%%%%%%%%%%%%%%%%%%%%%%%%%%%%%%%%%%%%%%%%%%%%%%%%%%%%%%%%%%%%%%
% This package provides theorem-like environments.                             %
%%%%%%%%%%%%%%%%%%%%%%%%%%%%%%%%%%%%%%%%%%%%%%%%%%%%%%%%%%%%%%%%%%%%%%%%%%%%%%%%

\usepackage{amsthm}

%%%%%%%%%%%%%%%%%%%%%%%%%%%%%%%%%%%%%%%%%%%%%%%%%%%%%%%%%%%%%%%%%%%%%%%%%%%%%%%%
% Theorem environments, use "thm" for italic fonts inside theorem environments %
% and "def" for normal fonts. These differ from the originals: newline is      %
% created after the complete name (including optional parenthesis) of the      %
% theorem.                                                                     %
%%%%%%%%%%%%%%%%%%%%%%%%%%%%%%%%%%%%%%%%%%%%%%%%%%%%%%%%%%%%%%%%%%%%%%%%%%%%%%%%
 
\newtheoremstyle{thm}       % name
    {15pt}                  % Space above
    {10pt}                  % Space below
    {\itshape}              % Body font
    {}                      % indent amount
    {\bf}                   % Theorem head font
    {}                      % Punctuation after theorem head
    {\newline}              % space after theorem head
    {}                      % Theorem head spec

\newtheoremstyle{def}
    {15pt}
    {10pt}
    {}
    {}
    {\bf}
    {}
    {\newline}
    {}

%%%%%%%%%%%%%%%%%%%%%%%%%%%%%%%%%%%%%%%%%%%%%%%%%%%%%%%%%%%%%%%%%%%%%%%%%%%%%%%%
% The following commands provide enumeration features that yield a section-    %
% based numbering, which is continous through all defined environments. This   %
% is achieved by anchoring one environment to the section numbering. All       %
% subsequent environment are numbered in the same fashion as the single one.   %
%%%%%%%%%%%%%%%%%%%%%%%%%%%%%%%%%%%%%%%%%%%%%%%%%%%%%%%%%%%%%%%%%%%%%%%%%%%%%%%%

\swapnumbers

\theoremstyle{thm}

\newtheorem{intTheorem}{Satz}[subsection]             % Theorem
\newtheorem{intCorollary}[intTheorem]{Korollar}    % Corollary
\newtheorem{intLemma}[intTheorem]{Lemma}           % Lemma

\theoremstyle{def}
\newtheorem{intDefinition}[intTheorem]{Definition} % Definition

%%%%%%%%%%%%%%%%%%%%%%%%%%%%%%%%%%%%%%%%%%%%%%%%%%%%%%%%%%%%%%%%%%%%%%%%%%%%%%%%
% Actual environments. The parameters are:                                     %
% #1 - name of the block (e.g. term to be defined)                             %
% #2 - label of the block (purely for internal reference purposes)             %
%%%%%%%%%%%%%%%%%%%%%%%%%%%%%%%%%%%%%%%%%%%%%%%%%%%%%%%%%%%%%%%%%%%%%%%%%%%%%%%%

\newenvironment{theorem}[2] % 1=Name, 2=Label
  {\begin{intTheorem}[#1] 
   \label{#2}
   %\addcontentsline{toc}{section}{\protect\numberline{\ref{#2}} #1}
   } 
  {\end{intTheorem}}

\newenvironment{corollary}[2] % 1=Name, 2=Label
  {\begin{intCorollary}[#1]
   \label{#2}
   %\addcontentsline{toc}{section}{\protect\numberline{\ref{#2}} #1}
   }
  {\end{intCorollary}}

\newenvironment{lemma}[2] % 1=Name, 2=Label
  {\begin{intLemma}[#1]
   \label{#2}
   %\addcontentsline{toc}{section}{\protect\numberline{\ref{#2}} #1}
   } 
  {\end{intLemma}}
 
\newenvironment{definition}[2] % 1=Name, 2=Label
  {\begin{intDefinition}[#1]
   \label{#2}
   %\addcontentsline{toc}{section}{\protect\numberline{\ref{#2}} {#1}}
   }
  {\end{intDefinition}}

\usepackage{enumerate}

\author{Name des Autors}

\title{Name der Arbeit}

\begin{document}

\maketitle

\begin{abstract}
In diesem Dokument wird das Grundger\"ust einer
Seminarausarbeitung vorgestellt.
Wir zeigen, wie man einige der vordefinierten Befehle unserer Vorlage verwendet
und geben eine grobe Strukturierung an.
\end{abstract}

\section{Einleitung}

Hier beschreiben wir in Worten die Problemstellung unserer Arbeit. Wir
versuchen dabei, Formeln u. \"a. zu vermeiden. Um das Beispiel zu
vervollst\"andigen wenden wir uns einem konkreten Problem zu -- wir zeigen
n\"amlich im Verlauf dieser Arbeit, wie man mit Hilfe des bekannten
Unendlichkeitsaxioms die Existenz der nat\"urlichen Zahlen erh\"alt und zeigen
exemplarisch auf, wie man die Arithmetik darauf erkl\"art. Im zweiten Kapitel
werden wir dazu den Begriff der induktiven Menge einf\"uhren und die Existenz
einer kleinsten induktiven Menge (bez\"uglich Inklusion) zeigen. Im dritten
Abschnitt gehen wir auf das Prinzip der vollst\"andigen Induktion ein und geben
eine Anwendung an.

\section{Die nat\"urlichen Zahlen}

Wir benutzen die Definition der nat\"urlichen Zahlen \"uber sogenannte
induktive Mengen.

\begin{definition}{Induktive Menge}{def:induktiveMenge}
Sei $\mathcal{I}$ eine Menge. $\mathcal{I}$ hei\ss{}t \textbf{induktiv}
\begin{align*}
:\!\iff \emptyset \in \mathcal{I}
        \, \wedge \,
        \forall\, A \in \mathcal{I}: A \cup \set{A} \in \mathcal{I}\text.
\end{align*}
\end{definition}

Die Existenz induktiver Mengen kann \"uber das Unendlichkeitsaxiom zugesichert
werden.

\begin{lemma}{Durchschnitte induktiver Mengen sind induktiv}{lem:durchschnitte}
Sei $\mathfrak{I}$ eine nichtleere Menge induktiver Mengen. Dann ist auch
$\bigcap\mathfrak{I}$ induktiv.
\end{lemma}
\begin{proof}
 Es gilt
zun\"achst:
\begin{align*}
    \emptyset \in \bigcap\mathfrak{I}
    & \iff
    \forall\, \mathcal{A} \in \mathfrak{I} : \emptyset \in \mathcal{A}
        \tag{Definition $\bigcap$}\\
    & \iff \text{true}
        \tag{siehe unten}
\end{align*}
Die Korrektheit des letzten Schrittes folgt aus der Tatsache, dass jedes
$\mathcal{A} \in \mathfrak{I}$ induktiv ist und somit $\emptyset \in
\mathcal{A}$ gilt. Sei nun $A \in \bigcap\mathfrak{I}$. Dann gilt:
\begin{align*}
    A \cup \set{A} \in \bigcap\mathfrak{I} 
    & \iff \forall\, \mathcal{A} \in \mathfrak{I}:
                     A \cup \set{A} \in \mathcal{A}
         \tag{Definition $\bigcap$}\\
    & \iff \forall\, \mathcal{A} \in \mathfrak{I}: \text{true}
         \tag{$\mathcal{A} \in \mathfrak{I} \Rightarrow \mathcal{A}$ induktiv}\\
    & \iff \text{true}
\end{align*}
Damit ist $\bigcap\mathfrak{I}$ eine induktive Menge.
\end{proof}


\begin{lemma}{Kleinste induktive Menge}{lem:kleinsteInduktiveMenge}
Es existiert eine induktive Menge $\mathcal{I}_0$ mit der Eigenschaft, dass
f\"ur jede induktive Menge $\mathcal{I}$ gilt:
$\mathcal{I}_0 \subseteq \mathcal{I}$.
\end{lemma}
\begin{proof}
Sei $\mathcal{M}$ eine induktive Menge (existiert nach dem
Unendlichkeitsaxiom). Wir definieren $\mathfrak{I} := \Set{ \mathcal{A}
\subseteq \mathcal{M} | \mathcal{A} \text{ induktiv}}$. Offensichtlich ist
$\mathfrak{I} \neq \emptyset$, da $\mathcal{M} \in \mathfrak{I}$ gilt. Damit
ist die folgende Setzung wohldefiniert:
\[\mathcal{I}_0 := \bigcap \mathfrak{I} \text.\]
Nach Lemma \ref{lem:durchschnitte} wissen wir bereits, dass $\mathcal{I}_0$
induktiv ist. Sei nun $\mathcal{I}$ eine weitere induktive Menge. Dann ist
wiederum nach Lemma \ref{lem:durchschnitte} die Menge $\mathcal{I} \cap
\mathcal{I}_0$ induktiv. Au\ss{}erdem ist $\mathcal{I} \cap
\mathcal{I}_0 \subseteq \mathcal{I}_0 \subseteq \mathcal{M}$, also $\mathcal{I}
\cap \mathcal{I}_0 \in \mathfrak{I}$. Damit gilt $\mathcal{I}_0 =
\bigcap\mathfrak{I} \subseteq \mathcal{I} \cap \mathcal{I}_0$ und somit
$\mathcal{I} \cap \mathcal{I}_0 = \mathcal{I}_0$, was \"aquivalent zu
$\mathcal{I}_0 \subseteq \mathcal{I}$ ist. Also ist $\mathcal{I}_0$ die
kleinste induktive Menge.
\end{proof}

\begin{definition}{Die nat\"urlichen Zahlen}{def:natuerlicheZahlen}
Wir definieren die Menge der nat\"urlichen Zahlen $\mathbb{N}$ als die nach dem
Lemma \ref{lem:kleinsteInduktiveMenge} existierende kleinste induktive Menge.
Im weiteren Verlauf werden wir die Elemente dieser Menge mit kleinen
lateinischen Buchstaben bezeichnen. Ferner definieren wir $0 := \emptyset$.
\end{definition}

\section{Arithmetik der nat\"urlichen Zahlen}

Nach der rein mengentheoretischen Einf\"uhrung widmen wir uns nun der
Definition von Rechenoperationen auf den nat\"urlichen Zahlen zu.

\begin{definition}{Nachfolgerfunktion}{def:nachfolgerfunktion}
Wir definieren:
\[\nu : \mathbb{N} \to \mathbb{N}, \, n \mapsto n \cup \set{n}\]
und nennen diese Funktion \textbf{die Nachfolgerfunktion auf den nat\"urlichen
Zahlen}.
\end{definition}

Die Idee bei dieser Funktion ist, dass f\"ur eine nat\"urliche Zahl $n \in
\mathbb{N}$ durch $\nu(n)$ der Nachfolger gegeben ist, den man sich anschaulich
als $n + 1$ vorstellen kann.

\begin{theorem}{Vollst\"andige Induktion (abstrakt)}
               {satz:vollstaendigeInduktion1}
Sei $A \subseteq \mathbb{N}$ mit den folgenden Eigenschaften:
\begin{enumerate}[(a)]
    \item $0 \in A$,
    \item $\forall\, n \in \mathbb{N}: n \in A \rightarrow \nu(n) \in A$.
\end{enumerate}
Dann gilt $A = \mathbb{N}$.
\end{theorem}
\begin{proof}
Die Eigenschaften (a), (b) besagen offensichtlich, dass $A$ eine induktive
Menge ist. Da $\mathbb{N}$ die kleinste induktive Menge ist, gilt $\mathbb{N}
\subseteq A$. Nach Voraussetzung gilt aber auch $A \subseteq \mathbb{N}$, also
$A = \mathbb{N}$.
\end{proof}

Die Idee an dem Satz ist folgende -- will man zeigen, dass etwas f\"ur alle
nat\"urlichen Zahlen gilt, so betrachtet man die Menge jener Elemente, f\"ur
welche die besagte Eigenschaft gilt. Von dieser Menge zeigt man dann, dass sie
die $0$ enth\"alt und mit jedem Element $n$ auch deren Nachfolger $\nu(n)$.

Wir wenden diesen Satz nun sofort in genau dieser Form an, um eine Darstellung
der nat\"urlichen Zahlen zu erhalten.

\begin{lemma}{Darstellung nat\"urlicher Zahlen}
             {lem:darstellung}
Sei $n \in \mathbb{N}$. Dann ist $n = 0$ oder es existiert $m \in \mathbb{N}$
mit der Eigenschaft $n = \nu(m)$.
\end{lemma}
\begin{proof}
Sei
$A:=\Set{n \in \mathbb{N} | n = 0 \vee \exists\, m \in \mathbb{N}: n = \nu(m)}$.
Klar ist $0 \in A$. Sei $n \in A$. Wir m\"ussen zeigen, dass $\nu(n) \in A$
gilt. Es existiert ein $m \in \mathbb{N}$ mit $\nu(n) = \nu(m)$ (n\"amlich
$m := n$). Damit ist aber $\nu(n) \in A$. Nach Satz
\ref{satz:vollstaendigeInduktion1} erhalten wir $A = \mathbb{N}$.
\end{proof}

\section{Diskussion und Ausblick}

Wir haben mit Hilfe rein mengentheoretischer Methoden und des
Un\-end\-lich\-keits\-ax\-ioms die nat\"urlichen Zahlen eingef\"uhrt. Mit Hilfe der
Nachfolgerfunktion haben wir dann das Induktionsprinzip bewiesen und als
Anwendung die Darstellung der nat\"urlichen Zahlen als $0$ oder als Nachfolger
einer anderen nat\"urlichen Zahl erhalten. Unter Verwendung der Injektivit\"at
der Funktion $\nu$ k\"onnen wir auch die Addition nat\"urlicher Zahlen
erkl\"aren, welche dann wiederum die Definition der Multiplikation und der
Potenzierung zul\"asst. Eigenschaften dieser arithmetischen Operationen
k\"onnen wir jedes Mal mit Hilfe des Induktionsprinzips beweisen.

Die vorgestellte Einf\"uhrung nat\"urlicher Zahlen ist nicht die einzig
naheliegende. In der Variante von Peano werden die nat\"urlichen Zahlen direkt
mit Hilfe der Nachfolgerfunktion eingef\"uhrt und Eigenschaften dieser Funktion
werden axiomatisch gefordert (siehe \cite{Peano}).

Weitere Hintergrundinformationen zur Einf\"uhrung nat\"urlicher Zahlen findet
man in \cite{Dedekind}. Wir verweisen weiterhin auf \cite{Zermelo}, worin die
Axiome der Mengenlehre vorgestellt und untersucht werden.

\bibliographystyle{plain}
\bibliography{Seminar}

\end{document}