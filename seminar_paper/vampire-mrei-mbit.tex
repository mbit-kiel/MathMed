\documentclass{article}

%%%%%%%%%%%%%%%%%%%%%%%%%%%%%%%%%%%%%%%%%%%%%%%%%%%%%%%%%%%%%%%%%%%%%%%%%%%%%%%%
% German language settings. Allows the use of umlauts.                         %
%%%%%%%%%%%%%%%%%%%%%%%%%%%%%%%%%%%%%%%%%%%%%%%%%%%%%%%%%%%%%%%%%%%%%%%%%%%%%%%%
\usepackage[utf8]{inputenc}
\usepackage[T1]{fontenc}
\usepackage[ngerman]{babel}

%%%%%%%%%%%%%%%%%%%%%%%%%%%%%%%%%%%%%%%%%%%%%%%%%%%%%%%%%%%%%%%%%%%%%%%%%%%%%%%%
% These packages provide fonts for the document, but may require manual        %
% installation. The usual procedure (on Linux) is the following:               %
% 1. Download the package files                                                %
% 2. Extract the files to ~/texmf/tex/latex/newpackage                         %
% 3. If never done before, run "texhash ~/texmf" in a terminal                 %
% 4. run "sudo texhash" in a terminal                                          %
%                                                                              %
% At least one of these fonts is most likely available on your system. If this %
% is not the case, you can simply comment all package imports to use the       %
% default font "Computer Modern Roman".                                        %
%%%%%%%%%%%%%%%%%%%%%%%%%%%%%%%%%%%%%%%%%%%%%%%%%%%%%%%%%%%%%%%%%%%%%%%%%%%%%%%%

% TeX Gyre Times 
\usepackage{tgtermes}

% Fourier Font
% \usepackage{fourier}

% A somewhat outdated Times alternative
% \usepackage{mathptmx}

% A Palatino substitute with adjusted line spreading.
% \usepackage{mathpazo}
% \linespread{1.05}

%%%%%%%%%%%%%%%%%%%%%%%%%%%%%%%%%%%%%%%%%%%%%%%%%%%%%%%%%%%%%%%%%%%%%%%%%%%%%%%%
% This package provides the useful commands \set and \Set, that can be used to %
% denote sets. The \set command uses rigid brackets, whereas the \Set command  %
% employs flexible brackets to enclose the set.                                %
% Examples:                                                                    %
%  A \setminus \set{a_0}                                                       %
%  \set{x \in A | x \le 4}                                                     %
%  \Set{\int_{\Omega} f d\mu | f \in \mathcal{L}_2(\Omega, \mathfrak{A}, \mu)} %
%%%%%%%%%%%%%%%%%%%%%%%%%%%%%%%%%%%%%%%%%%%%%%%%%%%%%%%%%%%%%%%%%%%%%%%%%%%%%%%%

\usepackage{braket}

%%%%%%%%%%%%%%%%%%%%%%%%%%%%%%%%%%%%%%%%%%%%%%%%%%%%%%%%%%%%%%%%%%%%%%%%%%%%%%%%
% Several packages for mathematical symbols.                                   %
%%%%%%%%%%%%%%%%%%%%%%%%%%%%%%%%%%%%%%%%%%%%%%%%%%%%%%%%%%%%%%%%%%%%%%%%%%%%%%%%

\usepackage{amsmath, amsfonts, amssymb}

%%%%%%%%%%%%%%%%%%%%%%%%%%%%%%%%%%%%%%%%%%%%%%%%%%%%%%%%%%%%%%%%%%%%%%%%%%%%%%%%
% Provides double-stroked black board letters. In general \mathbb{LETTER is    %
% preferrable, but this package also provides a double-stroked 1: mathds{1}.   %
%%%%%%%%%%%%%%%%%%%%%%%%%%%%%%%%%%%%%%%%%%%%%%%%%%%%%%%%%%%%%%%%%%%%%%%%%%%%%%%%

% \usepackage{dsfont}

%%%%%%%%%%%%%%%%%%%%%%%%%%%%%%%%%%%%%%%%%%%%%%%%%%%%%%%%%%%%%%%%%%%%%%%%%%%%%%%%
% This package provides theorem-like environments.                             %
%%%%%%%%%%%%%%%%%%%%%%%%%%%%%%%%%%%%%%%%%%%%%%%%%%%%%%%%%%%%%%%%%%%%%%%%%%%%%%%%

\usepackage{amsthm}

%%%%%%%%%%%%%%%%%%%%%%%%%%%%%%%%%%%%%%%%%%%%%%%%%%%%%%%%%%%%%%%%%%%%%%%%%%%%%%%%
% Theorem environments, use "thm" for italic fonts inside theorem environments %
% and "def" for normal fonts. These differ from the originals: newline is      %
% created after the complete name (including optional parenthesis) of the      %
% theorem.                                                                     %
%%%%%%%%%%%%%%%%%%%%%%%%%%%%%%%%%%%%%%%%%%%%%%%%%%%%%%%%%%%%%%%%%%%%%%%%%%%%%%%%
 
\newtheoremstyle{thm}       % name
    {15pt}                  % Space above
    {10pt}                  % Space below
    {\itshape}              % Body font
    {}                      % indent amount
    {\bf}                   % Theorem head font
    {}                      % Punctuation after theorem head
    {\newline}              % space after theorem head
    {}                      % Theorem head spec

\newtheoremstyle{def}
    {15pt}
    {10pt}
    {}
    {}
    {\bf}
    {}
    {\newline}
    {}

%%%%%%%%%%%%%%%%%%%%%%%%%%%%%%%%%%%%%%%%%%%%%%%%%%%%%%%%%%%%%%%%%%%%%%%%%%%%%%%%
% The following commands provide enumeration features that yield a section-    %
% based numbering, which is continous through all defined environments. This   %
% is achieved by anchoring one environment to the section numbering. All       %
% subsequent environment are numbered in the same fashion as the single one.   %
%%%%%%%%%%%%%%%%%%%%%%%%%%%%%%%%%%%%%%%%%%%%%%%%%%%%%%%%%%%%%%%%%%%%%%%%%%%%%%%%

\swapnumbers

\theoremstyle{thm}

\newtheorem{intTheorem}{Satz}[subsection]             % Theorem
\newtheorem{intCorollary}[intTheorem]{Korollar}    % Corollary
\newtheorem{intLemma}[intTheorem]{Lemma}           % Lemma

\theoremstyle{def}
\newtheorem{intDefinition}[intTheorem]{Definition} % Definition

%%%%%%%%%%%%%%%%%%%%%%%%%%%%%%%%%%%%%%%%%%%%%%%%%%%%%%%%%%%%%%%%%%%%%%%%%%%%%%%%
% Actual environments. The parameters are:                                     %
% #1 - name of the block (e.g. term to be defined)                             %
% #2 - label of the block (purely for internal reference purposes)             %
%%%%%%%%%%%%%%%%%%%%%%%%%%%%%%%%%%%%%%%%%%%%%%%%%%%%%%%%%%%%%%%%%%%%%%%%%%%%%%%%

\newenvironment{theorem}[2] % 1=Name, 2=Label
  {\begin{intTheorem}[#1] 
   \label{#2}
   %\addcontentsline{toc}{section}{\protect\numberline{\ref{#2}} #1}
   } 
  {\end{intTheorem}}

\newenvironment{corollary}[2] % 1=Name, 2=Label
  {\begin{intCorollary}[#1]
   \label{#2}
   %\addcontentsline{toc}{section}{\protect\numberline{\ref{#2}} #1}
   }
  {\end{intCorollary}}

\newenvironment{lemma}[2] % 1=Name, 2=Label
  {\begin{intLemma}[#1]
   \label{#2}
   %\addcontentsline{toc}{section}{\protect\numberline{\ref{#2}} #1}
   } 
  {\end{intLemma}}
 
\newenvironment{definition}[2] % 1=Name, 2=Label
  {\begin{intDefinition}[#1]
   \label{#2}
   %\addcontentsline{toc}{section}{\protect\numberline{\ref{#2}} {#1}}
   }
  {\end{intDefinition}}


\usepackage{enumerate}
\usepackage{wrapfig}
\usepackage{listings}
\usepackage{color}


\makeatletter %definition of classes for highlighting
\lst@InstallKeywords k{tfftypes}{tfftypesstyle}\slshape{tfftypesstyle}{}ld
\lst@InstallKeywords k{tfffunctions}{tfffunctionsstyle}\slshape{tfffunctionsstyle}{}ld
\lst@InstallKeywords k{formulatypes}{formulatypesstyle}\slshape{formulatypesstyle}{}ld
\lst@InstallKeywords k{tptpkeywords}{tptpkeywordsstyle}\slshape{tptpkeywordsstyle}{}ld
\lst@InstallKeywords k{roles}{rolesstyle}\slshape{rolesstyle}{}ld
\makeatother

% Define Language for highlighting
\lstdefinelanguage{tptp}
{
	% list of keywords
	tfftypes={
		\$tType,
		\$i,
		\$o,
		\$int,
		\$rat,
		\$real,
		\$array1,
		\$array2,
	},
	formulatypes={
		fof,
		tff,
		cnf,
		tpi,
		thf,
		bnf,
	},
	tfffunctions={
		\$sum,
		\$product,
		\$difference,
		\$uminus,
		\$torat,
		\$toreal,
		\$less,
		\$lesseq,
		\$greater,
		\$greatereq,
	},
	tptpkeywords={
		\$ite\_f,
		\$ite\_t,
		\$let\_tt,
		\$let\_tf,
		\$let\_ft,
		\$let\_ff,
	},
	roles={
		axiom,
		hypothesis,
		conjecture,
		type,
	},	
	keywords={ %add more for highlighting of in formula defined functions
		member, 
		sum,
	},
	sensitive=true, % keywords are case-sensitive
	morecomment=[l]{//}, % l is for line comment
	morecomment=[s]{/*}{*/}, % s is for start and end delimiter
	morestring=[b]" % defines that strings are enclosed in double quotes
}

% Define Colors for highlighting
\usepackage{color}
\definecolor{blue}{RGB}{0,0,205}
\definecolor{lightblue}{RGB}{99,184,255}
\definecolor{green}{RGB}{0,139,0}
\definecolor{purple}{RGB}{139,0,139}
\definecolor{red}{RGB}{205,38,38}
\definecolor{orange}{RGB}{205,102,0}
\definecolor{brown}{RGB}{139,69,19}

% Set Language for highlighing
\lstset{
	language={tptp},
	basicstyle=\small\ttfamily, % Global Code Style
	captionpos=b, % Position of the Caption (t for top, b for bottom)
	extendedchars=true, % Allows 256 instead of 128 ASCII characters
	tabsize=2, % number of spaces indented when discovering a tab 
	columns=fixed, % make all characters equal width
	keepspaces=true, % does not ignore spaces to fit width, convert tabs to spaces
	showstringspaces=false, % lets spaces in strings appear as real spaces
	breaklines=true, % wrap lines if they don't fit
	numbers=left, % show line numbers at the left
	numberstyle=\tiny\ttfamily, % style of the line numbers
	%commentstyle=\color{missing_color}, % style of comments
	%stringstyle=\color{missing_color}, % style of strings
	tfftypesstyle=\color{lightblue}, % style of tff-types (e.g. $int)
	formulatypesstyle=\color{purple}, % style of foruma-types (e.g. fof)
	tfffunctionsstyle=\color{orange}, % style of tff-functions (e.g. $sum)
	tptpkeywordsstyle=\color{green}, % style of tptp-keywords (e.g. $ite_f)
	rolesstyle=\color{brown}, % style of roles (e.g. axiom)
	keywordstyle=\color{blue}, %style of general keywords like introduced functions (e.g. member)
}

\author{
	Maximilian Reinhart\\
	Martin Bittermann
}

\title{VAMPIRE: Grundlagen und Anwendung}

% Generate PDF hyperlinks when referencing sections and stuff. 
\usepackage{hyperref}

% Frakturzeichen -  Double-stroke symbols
\usepackage{dsfont}

\begin{document}

\maketitle

%%%%%%%%%%%%%%%%%%%%%%%%%%%%%%%%%%%%%%%%%%%%%%%%%%%%%%%%%%%%%%%%%%%%%%%%%%
\begin{abstract}
	%Notes:
	%Klaren Bezug auf [1] herausstellen.
	%Grafiken nicht am Anfang des paper, eher am anfang von kapitel 2.
	Dieser Artikel verschafft zunächst ein grundlegendes Verständnis über den automatischen Theorembeweiser VAMPIRE und
	stellt den Bezug zu ~\cite{cav2013} her. \\
	Es wird eine Einführung in die generelle Benutzung der Software und die verwendete Problembeschreibungssprache, TPTP, gegeben, 
	die Funktionsweise und einige ausgewählte Algorithmen erklärt und
	die Ausgabe von VAMPIRE sowohl analysiert als auch grafisch dargestellt.
	Es werden Parallelen zu anderen Theorembeweisern aufgezeigt, VAMPIRE mit diesen Verglichen 
	und zum Schluss wird mit Hinblick auf mögliche Anwendungsbereiche ein Fazit und ein Ausblick gegeben.
\end{abstract}


%%%%%%%%%%%%%%%%%%%%%%%%%%%%%%%%%%%%%%%%%%%%%%%%%%%%%%%%%%%%%%%%%%%%%%%%%%
\section{Einleitung}
\label{sec:introduction}


Dieser Artikel bezieht sich primär auf ~\cite{cav2013} und versucht ebenso einen grundlegenden Überblick zu geben.
Dazu beziehen wir uns auf die Artikel ~\cite{hoder2010} und ~\cite{kovacs2009finding}.
(Überblick) 

(Struktur dieses Artikel mit Verweisen auf Kapitel)

Zuerst werden \hyperref[sec:foundations]{Grundlagen} über ATP vermittelt.\\
Dann \hyperref[sec:input]{TPTP Eingabe}.\\
Danach wird in \hyperref[sec:invocation]{Benutzung von Vampire}.\\
Anschließend \hyperref[sec:mechanics]{Funktionsweise}.\\
Schließlich \hyperref[sec:output]{Beweisausgabe} erklärt.\\
Schlussendlich \hyperref[sec:conclusion]{Fazit}.

%%%%%%%%%%%%%%%%%%%%%%%%%%%%%%%%%%%%%%%%%%%%%%%%%%%%%%%%%%%%%%%%%%%%%%%%%%
\section{Grundlagen}
\label{sec:foundations}
Automatische Theorembeweiser (ATP) versuchen ohne Benutzerinteraktion Aussagen über gegebene Formeln durch Anwendung von Regeln und Axiomen, die ihnen vorliegen, zu beweisen.
Hierfür wird das Problem von Hand in eine formale Struktur gebracht, die der Beweiser verarbeiten kann.
Diese Formalisierung ist auch für triviale Probleme nicht gleichermaßen einfach und ist daher eine gefährliche Fehlerquelle. 
Nichtsdestotrotz ist für komplexe Probleme das Formalisieren einfacher als das Beweisen per Hand und nach korrekter Formalisierung liefert der ATP, 
sofern er einen Beweis findet, einen, wovon auszugehen ist, richtigen Beweis. 
Anwendung finden ATP in vielerlei Gebieten, unter anderem Hardware-Design und -verifikation, Softwareverifikation, wissensbasierte Systeme und klassische mathematische Problemstellungen.
Der Vorteil von ATP liegt in diesen Gebieten in der Fähigkeit, Probleme zu bearbeiten, 
die für eine manuelle Beweisführung (teils um Größenordnungen) zu umfangreich sind.

VAMPIRE ist ein ATP für Logik erster Stufe (Prädikatenlogik) mit Gleichheit. Die erste Implementierung stammt von 1993, im Verlauf wurde die Software mehrfach um- und neugeschrieben.
Die Software umfasst heute etwa 152,000 SLOC in C++ und wurde hauptsächlich von Andrei Voronkov und Kry{\v{s}}tof Hoder an der University of Manchester entwickelt.
Neuere Entwicklungen und Ideen stammen vemehrt von Laura Kov{\'a}cs und Ioan Dragan, der Arbeiten am SAT-Beweiser und der Bound Propagation übernommen hat.
Derzeit wird an der vierten Version von VAMPIRE gearbeitet.

VAMPIRE hat dreißig Preise gewonnen und seit 1999 in der Weltmeisterschaft für automatische Theorembeweiser (CASC) jedes Mal mindestens in einer Kategorie gewonnen.
Insgesamt zwölf Mal hat er bei der CASC die Kategorie für Logik erster Stufe gewonnen und elf Mal für CNF/MIX. ~\cite{vampirehp} \\ Daher gilt VAMPIRE allgemein hin als sehr schnell.
Dank seiner Multiplattform- und Multicore-Unterstützung soll er auf allen gängigen Betriebssystemen, wie Linux, Windows und Mac eingesetzt werden können und mehrere Beweisversuche gleichzeitig bearbeiten können.
Wir vermögen zu bestätigen, dass die uns vorliegende Version 3.0 auf Windows 7, 64-bit funktioniert, lediglich mehrere Kerne wurden nicht benutzt und somit nicht mehrere Beweisversuche parallel unternommen.
Vampire löst Probleme über die ihm über die Kommandozeile mitgegebene Strategie, die über den Mode-Befehl eingestellt werden kann und besitzt eine auf einzigartige limited resource strategy, die für kurze Zeitlimits sehr effizient sein soll. Mit Vampire ist es, neben seiner Funktion als ATP, auch möglich ein Expertensystem zu betreiben, das auf Grund von gegebenen Aussagen Antworten aus einer Wissenssammlung geben kann, ebenso ist VAMPIRE zur Programmverifikation für C-Programme mit Schleifen geeignet. 
Leider steht Vampire entgegen der Aussage in ~\cite{cav2013} nicht unter einer liberalen Lizenz, sondern unter einer nicht-freien, proprietären Lizenz, die in ~\cite{vampirehp} einzusehen ist.

%%%%%%%%%%%%%%%%%%%%%%%%%%%%%%%%%%%%%%%%%%%%%%%%%%%%%%%%%%%%%%%%%%%%%%%%%%
\section{TPTP Eingabe}
\label{sec:input}

TPTP ist die Sprache, in der Probleme und Axiome geschrieben sein müssen, damit Vampire sie bearbeiten kann. TPTP besitzt Ähnlichkeit zu Prolog und wird von vielen heutigen Therembeweisern für Logik 1. Stufe unterstützt. \cite[S. 4]{cav2013}. Die Sprache dient als Grundlage für die gleichnamige \textit{Thousands of Problems for Theorem Provers (TPTP)} Bibliothek.\\
Die Sprache unterstützt folgende Formeltypen:
\begin{itemize}
	\item First order formula (FOF)
	\item Typed first order formula (TFF)
	\item Clause normal form (CNF)
	\item TPTP Process Instruction Language  (TPI)
	\item Typed higher order form (THF)
\end{itemize}
Vampire unterstützt hieraus FOF, TFF und CNF.

\subsection{Syntax}
\label{subsec:syntax}


\begin{wraptable}{r}{4cm}
\begin{tabular}{|c|c|}
	\hline TPTP & Bedeutung \\ 
	\hline ?[X] & $\exists$X \\
	\hline ![X] & $\forall$X \\
	\hline = & = \\
	\hline != & $\neq$ \\
	\hline $\sim$ & $\lnot$ \\
	\hline / & $\lor$ \\
	\hline \& & $\land$ \\
	\hline => & $\Rightarrow$ \\
	\hline <=> & $\leftrightarrow$ \\
	\hline Y & Variable \\
	\hline y & Literal \\
	\hline f(x, A, B) & Prädikat \\
	\hline \% \dots & Kommentar \\
	\hline
\end{tabular} 
\end{wraptable}

Die Syntax von TPTP ist wie folgt definiert:\\
Eine Formel in TPTP beginnt stets mit der Information über ihren Typ. Hierbei können zum Beispiel $fof(\dots)$, $tff(\dots)$, $cnf(\dots)$ und einige mehr verwendet werden.
Innerhalb der Klammern werden drei Bereiche unterschieden, die jeweils mit einem Komma voneinander getrennt sind. 
Zuerst kommt der Name der Formel, welcher lediglich zur besseren Übersicht im späteren Beweis verwendet wird und ansonsten weder benutzt noch eindeutig sein muss,
dann die Rolle, die diese Formel einnimmt, hierbei kann es sich zum Beispiel um ein Axiom (axiom), eine Hypothese (hypothesis) oder auch eine Behauptung (conjecture) handeln.
Eine Rolle ist eine wichtige Voraussetzung für den Beweiser, denn hierdurch erfährt er, welche Grundvoraussetzungen und Annahmen getroffen wurden, die er als richtig behandeln kann um die gestellte Behauptung zu beweisen.
Letztendlich steht die Formel in der Anfangs gewählten Syntax, die der Beweiser behandeln kann.



\subsection{first oder formular}
\label{subsec:tptpfof}
Wollen wir nun die Formel 
\[\forall X,A : X \in \bigcup A \Leftrightarrow \exists Y : Y \in A \land X \in Y,\]
die für einen späteren Beweis als Axiom dienen soll, in TPTP umwandeln, 
so bedarf es der Definition der Operationen, die im formalen Aufschrieb einfach benutzt werden können, 
die in der Syntax von TPTP jedoch nicht vorhanden sind.
Hierzu gehört neben $\in$ auch  $\bigcup$. 
Nun verhält es sich derart, dass wenn wir die fehlenden Operationen als Prädikate definieren, 
wir nicht zwingend eines zur Zeit in einem Axiom definieren, sondern manchmal allein durch die Definition und Nutzung eines Prädikates der Verwendungszweck erschlossen werden kann.
Ein Prädikat wird in VAMPIRE definiert indem es benutzt wird und im Verlauf des Beweises sucht VAMPIRE nach Möglichkeiten aus den gegebenen Axiomen und Hypothesen Schlüsse und Herleitungen zu ziehen.
So wandelt sich die obige Formel in TPTP um in \\ \\
\begin{lstlisting}[language=tptp]
! [X,A] : (member(X,sum(A)) 
	<=> ? [Y] : (member(Y,A) & member(X,Y)).
\end{lstlisting}
$\in$ wurde durch $member(\dots)$ ersetzt, $\bigcup$ durch $sum(\dots)$.
Legen wir jetzt fest, dass es sich um Logik erster Stufe handelt ($fof(\dots)$), dass die Definition, die Namensgeber wird, da wir diese vorwiegend tätigen, die große Vereinigung über eine Menge ist ($sum$) und, dass es sich um ein $Axiom$ handelt,
entsteht in Verbindung mit unser umgewandelten TPTP-Formel folgende fertige TPTP-Aussage:
\begin{lstlisting}[language=tptp]
fof(sum, axiom, (	
	! [X,A] : (member(X, sum(A)) 
		<=> ? [Y] : (member(Y, A) & member(X, Y))))).
\end{lstlisting}

Eine solcher Block wird immer mit einem Punkt abgeschlossen, dieser markiert das Ende.

Zusätzlich mit diesem und anderen Axiomen, die es nutzen, definiert sich das $member$ Prädikat automatisch.
Über den Befehl $include(\dots)$ können Dateien, wie zum Beispiel eine Datei mit vorgefertigten Axiomen, von extern in die TPTP-Problem-Datei eingebunden werden.

\subsection{typed first oder formular}
\label{subsec:tptptff}

In $tff(\dots)$ können getypte Variablen verwendet werden. Um eine Variable zu typen muss ihr bei ihrer Verwendung im Quantor mit einem Doppelpunkt ein Typenausdruck (sort) folgen.
Typenausdrücke sind \begin{itemize}
	\item \$i: individuell. Standard-Typ, wenn nichts angegeben
	\item \$o: Boolesch
	\item \$int: Integer
	\item \$rat: Rational
	\item \$real: Reell
	\item \$array1: Integer-Array
	\item \$array2: Array von Integer-Array
\end{itemize}
wobei die letzten beiden spezifisch von VAMPIRE definiert sind und nicht im TPTP-Standard verankert sind.
Eigene Typenausdrücke können durch ein ähnliches Schema definiert werden wie ein generelles $fof$-Formular:
\begin{lstlisting}[language=tptp]
tff(name, type, typ_name: $tType).
\end{lstlisting}
Ein Name kann wie bekannt frei gewählt werden und das Wort $type$ ist die Deklaration, dass es sich um eine Typendefinition handelt, wobei dieses Wort sowohl uneindeutig als auch redundant ist.
Zuletzt wird der $typ\_name$, der wieder frei gewählt werden kann, als neuer Typ definiert.

Beispiel:
\begin{lstlisting}[language=tptp]
tff(a_has_type_own,type,a : own).
tff(f_has_type_own,type,f : own * own > own).
\end{lstlisting}
Letzteres zeigt die Typdefinition einer Funktion f, die zwei Argumente übernimmt, deren Typ $own$ ist, und deren Ausgabe wieder des Typs $own$ ist.\\
\\
Um besser mit ganzen, rellen oder rationalen Zahlen arbeiten zu können, beinhaltet Vampire folgende, im TPTP-Standard definierten Funktionen:
\begin{itemize}
	\item $\$sum$: Addition $(x + y)$
	\item $\$product$: Multiplication $(x * y)$
	\item $\$difference$: Differnenz $(x - y)$
	\item $\$uminus$: einstelliges Minus $(-x)$
	\item $\$to rat$: Konvertierung zu rationalen Zahlen
	\item $\$to real$: Konvertierung zu reellen Zahlen
	\item $\$less$: kleiner als $(x < y)$
	\item $\$lesseq$: kleiner-gleich $(x \leq y)$
	\item $\$greater$: größer als $(x > y)$
	\item $\$greatereq$: größer-gleich $(x \geq y)$ \\ \\
\end{itemize}
Beispiel: \\ $(x + y) \geq 0$ mit x, y aus den ganzen Zahlen \\
\begin{lstlisting}[language=tptp]
tff(example,conjecture, 
	? [X:$int,Y:$int] : $greatereq($sum(X,Y),0)).\end{lstlisting}

\subsection{If-Then-Else}
\label{subsec:tptpitef}

Vereinfachend für die Programmverifikation kann beim umschreiben von Programmen nach TPTP das \\if-then-else-Konstrukt $ \$ite\_f $ für Formeln und $\$ite\_t$ für Terme benutzt werden.
\begin{lstlisting}[language=tptp]
$ite_f(Bedingung, Dann, Sonst)
\end{lstlisting}
(äquivalent für $\$ite\_t$).
\begin{lstlisting}[language=tptp]
tff(c1,axiom,((($ite_f(p & q, ~p | ~q, p & q)).

tff(c2,axiom,($ite_t(p,a,b) != (a & p)).
\end{lstlisting}
~\cite{hoder2011slides}
\subsection{Let...in}
\label{subsec:tptpletin}

Gleichsam kann das $let\dots in$-Konstrukt für für Therme $\$let\_tt$, für Formeln $\$let\_ff$, und zwischen Thermen und Formeln, bzw. Formeln und Thermen $\$let\_tf$ und $\$let\_ft$ verwendet werden.
\begin{lstlisting}[language=tptp]
tff(c1,axiom,
	$let_tt(f(X), g(X), f(a)) != g(a)).

tff(c2,axiom,
	$let_ff}(p(X), q(X) | r(X), p(c)) & ~q(c) & ~r(c)).

tff(c3,axiom,
	$let_tf(f(X), g(X), p(f(X))) & ~p(g(X))).

tff(c4,axiom,
$let_ft(p(X), q, $ite_t(p(a), a, b)) != $ite_t(q, a, b)).
\end{lstlisting}
%%%%%%%%%%%%%%%%%%%%%%%%%%%%%%%%%%%%%%%%%%%%%%%%%%%%%%
\subsection{Vergleich}
\label{subsec:tptpcomp}
TPTP und sein Struktur verhält sich ähnlich wie die der Logik-Programmiersprache Prolog, in der Prädikate durch ihre Benutzung und Angabe definiert werden.

Ähnlich wie bei PROVER 9 wird durch das angeben der Rolle der Verwendungszweck der Formel an VAMPIRE übergeben. Axiom und hypotheses sind dabei die assumtions und die conjecture sind die goals in PROVER 9.
Unterschiede in der Syntax sind selbstverständlich, da PROVER 9 nicht TPTP verwendet, obgleich es viele Übereinstimmungen gibt.
Auch der Punkt am Ende einer Zeile ist äquivalent. Unterschiedlich ist die Verwendung von Inifix-Operationen und deren vorherigen Definition als solche in PROVER 9, 
dies ist in VAMPIRE nicht möglich. In VAMPIRE ist es durch die $tff(\dots)$ dafür aber möglich explizit mit Variablen verschiedener Typen zu arbeiten.\cite{cav2013, prover9manual} 


%%%%%%%%%%%%%%%%%%%%%%%%%%%%%%%%%%%%%%%%%%%%%%%%%%%%%%
\section{Benutzung von Vampire}
\label{sec:invocation}

Grundsätzlich ist die Benutzung von VAMPIRE wie in \cite{cav2013} gesagt, recht einfach. Nachdem man das Programm vorliegen hat, wird über die Kommandozeile die enthaltene ausführbare Datei gestartet. Nun kann direkt in der Kommandozeile eine TPTP-Formel eingegeben werden, die an VAMPIRE übergeben werden soll. Falls diese schon vorgeschrieben in einer Datei vorliegt, kann diese Datei direkt über die üblichen Kommandozeilenbefehle an VAMPIRE übergeben werden. Nachdem VAMPIRE das Problem erhalten hat, beginnt es direkt mit der Arbeit.

$ vampire.exe < input.txt > output.txt $ beispielsweise liefert den Beweis für das in der input.txt geschriebene Problem und schreibt diesen, sonst in der Kommandozeile angezeigten Beweis, in die output.txt.

\subsection{Kommandozeilenparameter}
\label{subsec:commands}
VAMPIRE lässt sich hochgradig in seiner Bearbeitung des Problems beeinflussen. Hierzu werden dem Aufruf in der Konsole Parameter beigefügt von denen hier nur einige genannt werden.

\subsubsection{-~-theory\_axioms\_off}
\label{subsubsec: commandtheoryaxiomsoff}
Neben den Axiomen, die man der Problem-Datei hinzufügt, fügt VAMPIRE selbstständig im bekannte Axiome hinzu. Dies kann mit diesem Parameter unterbunden werden.
\subsubsection{-~-proof\_off}
\label{subsubsec: commandproofoff}
Möchte man die Gewissheit, dass das Problem bewiesen wurde, aber keinen Beweis an sich ausgegeben haben, kann man diesen Parameter übergeben, um die Ausgabe zu unterbinden.
\subsubsection{-~-t X oder -~-time\_limit X}
\label{subsubsec: commandtimelimit}
Mit diesem Parameter und X als ganze Zahl bricht VAMPIRE nach X Sekunden den Versuch ab das Problem zu lösen, sollte bis dahin kein Beweis gefunden worden sein.
In diesem Fall gibt VAMPIRE $time~limit~expired$ aus. Standardmäßig liegt das Zeitlimit bei 60  Sekunden.
\subsubsection{-~-m X oder -~-memory\_limit X}
\label{subsubsec: commandmemorylimit}
Ähnlich wie bei dem Zeitlimit wird hier mit X als ganzzahliger Wert das Speicherlimit in Megabyte für VAMPIRE. Grundsätzlich nutzt Vampire allen verfügbaren Speicher, den es adressieren kann, was bei 32-bit Binaries weniger als 4GB entspricht.
\subsubsection{-~-age\_weight\_ratio X:Y}
\label{subsubsec: commandageweightratio}
Das Verhältnis von Alter zu Gewicht bestimmt, von welchen der beiden Prioritätsqueues, die VAMPIRE von Klauseln unterhält, wie viele gewählt werden. X und Y müssen ganzzahlige, positive Zahlen sein.
2:3 würde beispielsweise bedeuten, dass insgesamt 5 Klauseln gewählt werden, davon 2 der ältesten und 3 der leichtesten. 
\subsubsection{-~-nongoal\_weight\_coefficient}
\label{subsubsec: commandweightcoefficient}
Erlaubt das einstellen des Koeffizienten, der das Gewicht einer Klausel bestimmt. Normalerweise ist die Anzahl der Symbole in der Klausel gleichsam das Gewicht.
\subsubsection{-~-max\_weight}
\label{subsubsec: commandmaxweight}
Stellt das ansonsten von VAMPIRE veränderbare maximale Klauselgewicht ein. VAMPIRE kann bei Speicherknappheit oder bei nutzen der $ limited~resource~strategy $ das Maximum ansonsten automatisch verringern um schwere Klauseln, zu verwerfen.
\subsubsection{-~-forced\_options}
\label{subsubsec: commandforcedoptions}
Alle hier einzeln aufgeführten Optionen können mit einem Doppelpunkt hintereinander gereiht werden um mehrfache Optionen gemeinsam zu forcieren. 
\begin{itemize}
	\item \textbf{propositional\_to\_bdd=off}\\
	Schaltet das Einführen von Prädikaten für BDD-Variablen aus. \\
	\item \textbf{splitting=off}\\
	Schaltet das Einführen von Prädikaten für Entscheidungspunkte aus. \\
	\item \textbf{equality\_proxy=off}\\
	\item \textbf{general\_splitting=off}\\
	\item \textbf{inequality\_splitting=0}\\
	\item \textbf{naming=0}\\
	Schaltet Einführen von Prädikaten zur Vermeidung von exponentiellem Ausdehnen während der Klausifizierung aus.
	Standardwert ist 8, je höher die Nummer, desto weniger eingeführte Namen.\\
\end{itemize}


\subsubsection{-~-saturation\_algorithm}
\label{subsubsec: commandssaturationalgorithm}
\begin{itemize}
	\item \textbf{lrs}\\
	Der standardmäßig gewählte Saturationsalgorithmus von VAMPIRE ist $lrs$, welcher sich auf die \\$ limited~resource~strategy $ bezieht.
	Dieser Algorithmus, der einzigartig in VAMPIRE sein soll, versucht unter den passiven und neuen Klauseln jene zu identifizieren, die niemals in der gegebenen Zeit gelöst werden können und verwirft diese. \cite{riazanov2003limited} \\
	\item \textbf{discount}\\
	Diese Art des Algorithmus wurde ursprünglich im DISCOUNT System von Jürgen Avenhaus, Jörg Denzinger und Matthias Fuchs implementiert.
	Einen ähnlich arbeitenden Algorithmus verwenden neben VAMPIRE auch WALDMEISTER, E und SPASS. 
	Die passiven Klauseln, der erzeugten Klauselmenge nehmen nicht am Inferenz- oder Vereinfachungssystem teil. \cite{riazanov2003limited} \\
	\item \textbf{otter}\\
	Dieser Art des Algorithmus wurde ursprünglich im OTTER Theorembeweiser von William McCune implementiert.
	Einen ähnlich arbeitenden Algorithmus verwenden neben VAMPIRE auch GANDALF und SPASS. 
	Anders als beim vorherig genannten Algorithmus ist die Teilnahme der passiven Klauseln an der Vereinfachung möglich, wie zum Beispiel, bei Gleichheit der Einheiten, das Neuschreiben oder die Subsumtion. \cite{riazanov2003limited} \\
\end{itemize}

\subsubsection{-~-modes}
\label{subsubsec:commandsmodes}

Modes bestimmen den Ablauf des Beweisprozesses indem eine spezielle Strategie oder Strategiegemische verwendet werden. 
Einige mögliche Modes wie in ~\cite{hoder2011slides} beschrieben sind:
\begin{itemize}
	\item \textbf{casc} \\
	Dieser auch bei dem gleichnamigen Wettbewerb verwendete Mode ist laut ~\cite{hoder2011slides}  der beste der möglichen Modes.
	Das gegebene Problem wird im Vorfeld analysiert um die Charakteristiken zu erkennen, dann wird es einer von derzeit 43 Klassen zugewiesen. Jede Klasse besitzt eine Abfolge von Strategien, die das Problem lösen sollten.
	Der Mode scheint aber in der von uns verwendeten Version von Vampire nicht lauffähig zu sein.\\
	\item \textbf{casc\_ltb} \\
	Dieser Mode wählt die Strategie äquivalent zum normalen casc-Mode. Der Input ist eine Batch-Datei nach den Vorgaben der CASC LTB (Large Theory Batch). Eine Besonderheit um Zeit zu sparen ist das nur einmalige einlesen der Axiome und das anschließende Hinzufügen dieser zu den Problemen denen sie zugewiesen sind. Dieser Mode ist multiprocessing-fähig und kann somit mehrere Strategien parallel verwendet.
	Der Mode scheint aber in der von uns verwendeten Version von Vampire nicht lauffähig zu sein.\\
	\item\textbf{ axiom\_selection} \\
	Sowohl Input als auch Output bei diesem Mode sind Formeln in TPTP oder CNF, wodurch er als Filter fungieren kann, da sein Output von zb. anderen Modes gelesen werden kann.
	Er vollzieht eine Sine-Axiom-Selection, wie in ~\cite{sinquanon} beschrieben.\\
	\item \textbf{clausify} \\
	Dieser Mode konvertiert Formeln von TPTP nach CNF um und unterstützt dabei getypte Formeln, Arithmetik und auch Antwortliterale. Er erlaubt die Anwendung von etlichen Preprocessing rules die Vampire beinhaltet. \\
	\item \textbf{grounding} \\
	Mit diesem Mode wird ein Einstein-Podolsky-Rosen-Paradoxon in ein aussagenlogisches Problem umgewandelt. Der Input erfolgt über TPTP, der Output ist im DIMACS CNF Format.
	Es wird splitting angewendet um die Anzahl der Variablen in den Klauseln und damit die der generierten aussagenlogischen Klauseln gering zu halten. \\
	\item \textbf{consequence\_elemination} \\
	Dieser Mode versucht aus einer gegebenen Menge an Behauptungen, der eine Theorie begründet liegen mag, eine der Behauptung aus den anderen herzuleiten.\\
	\item \textbf{bpa} \\
	Bound Propagation (bpa) ersetzt hierbei für das Lösen von lineare Ungleichheiten (quantorenfreie lineare, reelle und rationale Arithmetik) das Superpositionscalculus.
	\item \textbf{program\_analysis} \\
	Der letzte hier gezeigte Mode ist für die mit VAMPIRE mögliche Programmverifikation von C-Programmen. Hierbei muss die *.c-Datei an VAMPIRE übergeben werden.
\end{itemize}



%%%%%%%%%%%%%%%%%%%%%%%%%%%%%%%%%%%%%%%%%%%%%%%%%%%%%%
%\subsection{Segfaults}
%\label{subsec:segfaults}

%\begin{itemize}
%\item parentheses in input not aligned to magic waves in room
%\item it crash
%\end{itemize}



%%%%%%%%%%%%%%%%%%%%%%%%%%%%%%%%%%%%%%%%%%%%%%%%%%%%%%%%%%%%%%%%%%%%%%%%%%
\section{Funktionsweise}
\label{sec:mechanics}

%%%%%%%%%%%%%%%%%%%%%%%%%%%%%%%%%%%%%%%%%%%%%%%%%%%%%%
\subsection{Beweisverfahren}
\label{subsec:proofmech}

Dieser Abschnitt gibt einen Einblick in die beweisrelevanten Abläufe, die durch einen typischen Aufruf von VAMPIRE ausgelöst werden, bis zur Ausgabe eines Ergebnisses.


\subsection{Preprocessing}
\label{subsec:preprocessing}

Nachdem VAMPIRE die Eingabe gelesen hat, finden vor der eigentlichen, oft zeitintensiven
Beweissuche einige vorbereitende Schritte statt.

\begin{enumerate}
	\item Wähle eine relevante Teilmenge von Formeln \\(optional).
	\item Füge die Axiome hinzu (optional).
	\item Bereinige die Formeln.
	\item Sollte eine Formel $\top$ oder $\bot$ beinhalten, \\vereinfache sie.
	\item Entferne if-then-else und let-in Verbindungen.
	\item Ebne die Formeln.
	\item Entferne reine Prädikate.
	\item Entferne nicht genutzte Prädikatdefinitionen \\(optional).
	\item Konvertiere die Formeln in equivalence negation normal form (ennf).
	\item Ersetze Subformeln durch ihre Namen.
	\item Konvertiere die Formeln in negation normal form (nnf) (optional).
	\item Skolemisiere die Formeln.
	\item Ersetze equality axioms.
	\item Lege eine Ordnung der Literale fest.
	\item Transformiere die Formeln in \\conjunctive normal form (cnf).
	\item Entferne die Funktionsdefinitionen (optional).
	\item Wende inequality splitting an (optional).
	\item Entferne Tautologien.
	\item Entferne reine Literale (optional).
	\item Entferne Klauseldefinitionen (optional).
\end{enumerate}

\subsection{Beweis durch Widerspruch}
\label{subsec:refutation}

Vampire beweist Theoreme durch Widerspruch.
Dazu wird die Negation der Hypothese $F$, zusammen mit den Axiomen,
der Formelmenge hinzugefügt.
Nun leitet Vampire durch Inferenz neue Formeln aus den Axiomen und $F$ her, 
mit dem Ziel, die leere Klausel $\bot$ (\texttt{\$false} in TPTP-Schreibweise) zu folgern.
Kann diese hergeleitet werden, ist die gesamte Formelmenge unerfüllbar.
Demzufolge hat die \textit{Negation} der Hypothese unter den angegebenen Axiomen keine Gültigkeit,
im Umkehrschluss ist die Hypothese $F$ bewiesen. \cite[S. 5]{cav2013}

Ist die Formelmenge hingegen erfüllbar, kann $\neg F$ aus den Axiomen bewiesen werden.
Damit ist $F$ widerlegt. \cite[S. 7]{cav2013}

%%%%%%%%%%%%%%%%%%%%%%%%%%%%%%%%%%%%%%%%%%%%%%%%%%%%%%
\subsection{Inferenzsystem}
\label{subsec:inference}
Resolution and superposition inference system

%%%%%%%%%%%%%%%%%%%%%%%%%%%%%%%%%%%%%%%%%%%%%%%%%%%%%%
\subsection{Saturation}
\label{subsec:saturation}


\begin{definition}{Saturierte Formelmenge}{def:saturatedset}
	Eine Formelmenge $S$ heißt saturiert unter einem Inferenzsystem $\mathds{I}$, wenn für jede $\mathds{I}$-Inferenz mit Prämissen in $S$ die Konklusion auch in $S$ ist.
\end{definition}



%%%%%%%%%%%%%%%%%%%%%%%%%%%%%%%%%%%%%%%%%%%%%%%%%%%%%%
\subsection{Ausgewählte Algorithmen im Detail}
\label{subsec:algos}


%%%%%%%%%%%%%%%%%%%%%%%%%%%%%%%%%%%%%%%%%%%%%%%%%%%%%%%%%%%%%%%%%%%%%%%%%%
\section{Beweisausgabe}
\label{sec:output}

Während der Formelerzeugung bekommt jede Formel eine fortlaufende Nummer, beginnend mit den Axiomen und den Eingabeformeln. Wird ein Beweis gefunden, so gibt es einen gerichteten azyklischen Graphen (\textit{directed acyclic graph, DAG}) mit genau den Formeln als Knoten, welche zum Beweis (also dem Knoten \texttt{\$false}) hinführen. Die Kanten des DAG stellen Inferenz dar. Dieser DAG wird in der Beweisausgabe zeilenweise angezeigt. 
Am Beispiel dieser Zeile werden die einzelnen Bestandteile erklärt:\\ \\
\texttt{28. mult(inverse(X2),e) = X2 (2:6) [superposition 22,10]}
\begin{itemize}
	\item Formelnummer (\texttt{28.})\\
		Die eindeutige Nummer, die jeder Formel zugewiesen wird
	\item Konklusion (\texttt{mult(inverse(X2),e) = X2})\\
		Die Konklusionsformel in TPTP-Syntax. Hier können automatisch erzeugte und abgeänderte
		Variablen- und Literalnamen auftauchen.
	\item ??? (\texttt{(2:6)})\\
		Kein Plan wofür das steht.
	\item Resolutionsregel (\texttt{superposition})\\
		Die Regel, anhand derer diese Formel hergeleitet wurde.\\
		Formeln der Eingabe sind als \texttt{[input]} deklariert.
	\item Prämisse (\texttt{22,10})\\
		In diesem Resolutionsschritt benutzte Formeln.
		
	
\end{itemize}
In dieser Zeile ist also Formel 28 durch Superposition aus Formeln 22 und 10 hergeleitet worden.
Meist erzeugt Vampire viel mehr Formeln, die nicht zum Beweis führen und daher nicht zum Beweis-DAG gehören. Diese werden nicht ausgegeben.

%%%%%%%%%%%%%%%%%%%%%%%%%%%%%%%%%%%%%%%%%%%%%%%%%%%%%%
\subsection{Visualisierung}
\label{subsec:outputvis}

%%%%%%%%%%%%%%%%%%%%%%%%%%%%%%%%%%%%%%%%%%%%%%%%%%%%%%%%%%%%%%%%%%%%%%%%%%
\section{Fazit}
\label{sec:conclusion}
Write a small conclusion that summarizes what has been said ...?
Open research areas...


%%%%%%%%%%%%%%%%%%%%%%%%%%%%%%%%%%%%%%%%%%%%%%%%%%%%%%%%%%%%%%%%%%%%%%%%%%
% Bibliography

% The bibliography entries are stored in "vampire.bib"
\bibliographystyle{unsrt}
\bibliography{vampire}

\end{document}

