% Use the document class appropriate to your language and leave the other
% line commented out
\documentclass{acm_proc_article-sp-german}
%\documentclass{acm_proc_article-sp}

% These two lines help to keep enumerations and itemizations compact.
% Try commenting them out if you want to see the effect.
\usepackage{enumitem}
\setlist{nolistsep}

% The next two commands are for the code display example. Look into the
% documentation of the listings package for other configurations, in
% particular for a list of supported programming languages.
\usepackage{listings}
\lstset{language = C, 
        numbers=left, 
        numberstyle=\tiny,
        columns=fullflexible, 
        basicstyle=\sf, 
        xleftmargin = 0.5 cm}

% Display subfigures
\usepackage{subfig}

% Generate PDF hyperlinks when referencing sections and stuff. 
\usepackage{hyperref}

% Theorems like Definition and Proof
%\usepackage{amsthm}

% list bibliography by occurrence
\bibliographystyle{unsrt}

% Define Saturated sets
\newtheorem{satset}{Definition}

% This is the name of the folder you are placing your graphics files in.
% Defining it here makes LaTeX look for files there without you having
% to specify the folder again throughout the document.
\graphicspath{ {./graphics/} }

\begin{document}

\title{VAMPIRE: TODO: Titel}
\numberofauthors{2}
\author{
	\alignauthor
	Maximilian Reinhart\\
	\affaddr{mrei@informatik.uni-kiel.de}
	\alignauthor
	Martin Bittermann\\
	\affaddr{mbit@informatik.uni-kiel.de}
}

\date{\today}

\maketitle

%%%%%%%%%%%%%%%%%%%%%%%%%%%%%%%%%%%%%%%%%%%%%%%%%%%%%%%%%%%%%%%%%%%%%%%%%%
\begin{abstract}
%Notes:
%Klaren Bezug auf [1] herausstellen.
%Grafiken nicht am Anfang des paper, eher am anfang von kapitel 2.

Dieser Artikel verschafft zunächst ein grundlegendes Verständnis über den automatischen Theorembeweiser VAMPIRE und
stellt den Bezug zu [1] her. \\
Es wird eine Einführung in die grundlegende Benutzung der Software und die verwendete Problembeschreibungssprache, TPTP, gegeben. 
Danach werden die grundlegende Funktionsweise und einige ausgewählte Algorithmen erklärt, 
die Ausgabe von VAMPIRE analysiert und auch grafisch dargestellt.
Es werden Parallelen zu anderen Theorembeweisern aufgezeigt und Vergleiche gezogen.
Zum Schluss wird mit Hinblick auf mögliche Anwendungsbereiche ein Fazit und ein Ausblick gegeben.
\end{abstract}


%%%%%%%%%%%%%%%%%%%%%%%%%%%%%%%%%%%%%%%%%%%%%%%%%%%%%%%%%%%%%%%%%%%%%%%%%%
\section{Einleitung}
\label{sec:introduction}


Dieser Artikel bezieht sich primär auf ~\cite{cav2013} und versucht ebenso einen grundlegenden Überblick zu geben.
Dazu beziehen wir uns auf die Artikel ~\cite{hoder2010} und ~\cite{kovacs2009finding}.
(Überblick) 

(Struktur dieses Artikel mit Verweisen auf Kapitel)

Zuerst werden \hyperref[sec:foundations]{Grundlagen} über ATP vermittelt.\\
Dann \hyperref[sec:input]{TPTP Eingabe}.\\
Danach wird in \hyperref[sec:invocation]{Benutzung von Vampire}.\\
Anschließend \hyperref[sec:mechanics]{Funktionsweise}.\\
Schließlich \hyperref[sec:output]{Beweisausgabe} erklärt.\\
Schlussendlich \hyperref[sec:conclusion]{Fazit}.

%%%%%%%%%%%%%%%%%%%%%%%%%%%%%%%%%%%%%%%%%%%%%%%%%%%%%%%%%%%%%%%%%%%%%%%%%%
\section{Grundlagen}
\label{sec:foundations}

Grundlagen über ATP und etwas Einordnung von Vampire. Bezug auf und rücksichtslose Copypasta aus ~\cite{cav2013}.


%%%%%%%%%%%%%%%%%%%%%%%%%%%%%%%%%%%%%%%%%%%%%%%%%%%%%%%%%%%%%%%%%%%%%%%%%%
\section{TPTP Eingabe}
\label{sec:input}

TPTP Syntax und Beispiele.\\
TPTP Library.

\subsection{Syntax}
\label{subsec:syntax}

%%%%%%%%%%%%%%%%%%%%%%%%%%%%%%%%%%%%%%%%%%%%%%%%%%%%%%
\subsection{Vergleich}
\label{subsec:tptpcomp}
Vergleich zu anderen Problembeschreibungssprachen.


%%%%%%%%%%%%%%%%%%%%%%%%%%%%%%%%%%%%%%%%%%%%%%%%%%%%%%
\section{Benutzung von Vampire}
\label{sec:invocation}

Wichtige Informationen zur Benutzung in ~\cite{hoder2011slides}.

Hinzugefügt auch: ~\cite{kovacs2011slides} und ~\cite{hodervoronkov2012slides}, wobei dies nur Slides sind, die aber wichtige Informationen haben. Quelle hat Link ist aber unsichtbar im Literaturverzeichnis.

Streicht er bei dir das " aber " auch immer durch?

\subsection{Modes}
\label{subsec:modes}

Modes bestimmen den Ablauf des Beweisprozesses indem eine spezielle Strategie oder Strategiegemische verwendet werden. 
Dieser Befehl beginnt mit \"\-\-modes\" gefolgt von einem definierenden Schlüsselwort, welches den Mode bestimmt.
Einige mögliche Modes wie in ~\cite{hoder2011slides} beschrieben sind:
\begin{itemize}
\item casc \\
		Dieser auch bei dem gleichnamigen Wettbewerb verwendete Mode ist laut ~\cite{hoder2011slides}  der beste der möglichen Modes.
		Das gegebene Problem wird im Vorfeld analysiert um die Charakteristiken zu erkennen, dann wird es einer von derzeit 43 Klassen zugewiesen. Jede Klasse besitzt eine Abfolge von Strategien, die das Problem lösen sollten.
		Der Mode scheint aber in der von uns verwendeten Version von Vampire nicht lauffähig zu sein.\\
\item casc\_ltb \\
		Dieser Mode wählt die Strategie äquivalent zum normalen casc-Mode. Der Input ist eine Batch-Datei nach den Vorgaben der CASC LTB (Large Theory Batch). Eine Besonderheit um Zeit zu sparen ist das nur einmalige einlesen der Axiome und das anschließende Hinzufügen dieser zu den Problemen denen sie zugewiesen sind. Dieser Mode ist multiprocessing-fähig und kann somit mehrere Strategien parallel verwendet.
		Der Mode scheint aber in der von uns verwendeten Version von Vampire nicht lauffähig zu sein.\\
\item axiom\_selection \\
		Sowohl Input als auch Output bei diesem Mode sind Formeln in TPTP oder CNF, wodurch er als Filter fungieren kann, da sein Output von zb. anderen Modes gelesen werden kann.
		Er vollzieht eine Sine-Axiom-Selection, wie in ~\cite{sinquanon} beschrieben.\\
\item clausify \\
		Dieser Mode konvertiert Formeln von TPTP nach CNF um und unterstützt dabei getypte Formeln, Arithmetik und auch Antwortliterale. Er erlaubt die Anwendung von etlichen Preprocessing rules die Vampire beinhaltet. \\
\item grounding \\
		Mit diesem Mode wird ein Einstein-Podolsky-Rosen-Paradoxon in ein aussagenlogisches Problem umgewandelt. Der Input erfolgt über TPTP, der Output ist im DIMACS CNF Format.
		Es wird splitting angewendet um die Anzahl der Variablen in den Klauseln und damit die der generierten aussagenlogischen Klauseln gering zu halten. \\
\item consequence\_elemination \\
		Der letzte hier gezeigte Mode versucht aus einer gegebenen Menge an Behauptungen, der eine Theorie begründet liegen mag, eine der Behauptung aus den anderen herzuleiten.

\end{itemize}



%%%%%%%%%%%%%%%%%%%%%%%%%%%%%%%%%%%%%%%%%%%%%%%%%%%%%%
\subsection{Segfaults}
\label{subsec:segfaults}

\begin{itemize}
\item parentheses in input not aligned to magic waves in room
\item it crash
\end{itemize}



%%%%%%%%%%%%%%%%%%%%%%%%%%%%%%%%%%%%%%%%%%%%%%%%%%%%%%%%%%%%%%%%%%%%%%%%%%
\section{Funktionsweise}
\label{sec:mechanics}
So funktionsierts.

\subsection{Preprocessing}
\label{subsec:preprocessing}

%%%%%%%%%%%%%%%%%%%%%%%%%%%%%%%%%%%%%%%%%%%%%%%%%%%%%%
\subsection{Beweisverfahren}
\label{subsec:proofmech}

\begin{satset}
	Saturated set of S: Contains S, no more formulas can be inferred under the inference system.\\
	TODO: More math symbols!
\end{satset}

%%%%%%%%%%%%%%%%%%%%%%%%%%%%%%%%%%%%%%%%%%%%%%%%%%%%%%
\subsection{Ausgewählte Algorithmen im Detail}
\label{subsec:algos}


%%%%%%%%%%%%%%%%%%%%%%%%%%%%%%%%%%%%%%%%%%%%%%%%%%%%%%%%%%%%%%%%%%%%%%%%%%
\section{Beweisausgabe}
\label{sec:output}


%%%%%%%%%%%%%%%%%%%%%%%%%%%%%%%%%%%%%%%%%%%%%%%%%%%%%%
\subsection{Erklärung}
\label{subsec:outputexplained}

\begin{itemize}
\item Formelnummer
\item Formel
\item ???
\item Regel
\item Benutzte Formeln

\end{itemize}

%%%%%%%%%%%%%%%%%%%%%%%%%%%%%%%%%%%%%%%%%%%%%%%%%%%%%%
\subsection{Visualisierung}
\label{subsec:outputvis}

%%%%%%%%%%%%%%%%%%%%%%%%%%%%%%%%%%%%%%%%%%%%%%%%%%%%%%%%%%%%%%%%%%%%%%%%%%
\section{Fazit}
\label{sec:conclusion}
Write a small conclusion that summarizes what has been said ...?
Open research areas...

%%%%%%%%%%%%%%%%%%%%%%%%%%%%%%%%%%%%%%%%%%%%%%%%%%%%%%%%%%%%%%%%%%%%%%%%%%
% Bibliography

% The bibliography entries are stored in "myrefs.bib"
\bibliographystyle{abbrv}
\bibliography{myrefs}

\end{document}

