%%%%%%%%%%%%%%%%%%%%%%%%%%%%%%%%%%%%%%%%%%%%%%%%%%%%%%%%%%%%%%%%%%%%%%%%%%%%%%%%
% German language settings. Allows the use of umlauts.                         %
%%%%%%%%%%%%%%%%%%%%%%%%%%%%%%%%%%%%%%%%%%%%%%%%%%%%%%%%%%%%%%%%%%%%%%%%%%%%%%%%
\usepackage[utf8]{inputenc}
\usepackage[T1]{fontenc}
\usepackage[ngerman]{babel}

%%%%%%%%%%%%%%%%%%%%%%%%%%%%%%%%%%%%%%%%%%%%%%%%%%%%%%%%%%%%%%%%%%%%%%%%%%%%%%%%
% These packages provide fonts for the document, but may require manual        %
% installation. The usual procedure (on Linux) is the following:               %
% 1. Download the package files                                                %
% 2. Extract the files to ~/texmf/tex/latex/newpackage                         %
% 3. If never done before, run "texhash ~/texmf" in a terminal                 %
% 4. run "sudo texhash" in a terminal                                          %
%                                                                              %
% At least one of these fonts is most likely available on your system. If this %
% is not the case, you can simply comment all package imports to use the       %
% default font "Computer Modern Roman".                                        %
%%%%%%%%%%%%%%%%%%%%%%%%%%%%%%%%%%%%%%%%%%%%%%%%%%%%%%%%%%%%%%%%%%%%%%%%%%%%%%%%

% TeX Gyre Times 
\usepackage{tgtermes}

% Fourier Font
% \usepackage{fourier}

% A somewhat outdated Times alternative
% \usepackage{mathptmx}

% A Palatino substitute with adjusted line spreading.
% \usepackage{mathpazo}
% \linespread{1.05}

%%%%%%%%%%%%%%%%%%%%%%%%%%%%%%%%%%%%%%%%%%%%%%%%%%%%%%%%%%%%%%%%%%%%%%%%%%%%%%%%
% This package provides the useful commands \set and \Set, that can be used to %
% denote sets. The \set command uses rigid brackets, whereas the \Set command  %
% employs flexible brackets to enclose the set.                                %
% Examples:                                                                    %
%  A \setminus \set{a_0}                                                       %
%  \set{x \in A | x \le 4}                                                     %
%  \Set{\int_{\Omega} f d\mu | f \in \mathcal{L}_2(\Omega, \mathfrak{A}, \mu)} %
%%%%%%%%%%%%%%%%%%%%%%%%%%%%%%%%%%%%%%%%%%%%%%%%%%%%%%%%%%%%%%%%%%%%%%%%%%%%%%%%

\usepackage{braket}

%%%%%%%%%%%%%%%%%%%%%%%%%%%%%%%%%%%%%%%%%%%%%%%%%%%%%%%%%%%%%%%%%%%%%%%%%%%%%%%%
% Several packages for mathematical symbols.                                   %
%%%%%%%%%%%%%%%%%%%%%%%%%%%%%%%%%%%%%%%%%%%%%%%%%%%%%%%%%%%%%%%%%%%%%%%%%%%%%%%%

\usepackage{amsmath, amsfonts, amssymb}

%%%%%%%%%%%%%%%%%%%%%%%%%%%%%%%%%%%%%%%%%%%%%%%%%%%%%%%%%%%%%%%%%%%%%%%%%%%%%%%%
% Provides double-stroked black board letters. In general \mathbb{LETTER is    %
% preferrable, but this package also provides a double-stroked 1: mathds{1}.   %
%%%%%%%%%%%%%%%%%%%%%%%%%%%%%%%%%%%%%%%%%%%%%%%%%%%%%%%%%%%%%%%%%%%%%%%%%%%%%%%%

% \usepackage{dsfont}

%%%%%%%%%%%%%%%%%%%%%%%%%%%%%%%%%%%%%%%%%%%%%%%%%%%%%%%%%%%%%%%%%%%%%%%%%%%%%%%%
% This package provides theorem-like environments.                             %
%%%%%%%%%%%%%%%%%%%%%%%%%%%%%%%%%%%%%%%%%%%%%%%%%%%%%%%%%%%%%%%%%%%%%%%%%%%%%%%%

\usepackage{amsthm}

%%%%%%%%%%%%%%%%%%%%%%%%%%%%%%%%%%%%%%%%%%%%%%%%%%%%%%%%%%%%%%%%%%%%%%%%%%%%%%%%
% Theorem environments, use "thm" for italic fonts inside theorem environments %
% and "def" for normal fonts. These differ from the originals: newline is      %
% created after the complete name (including optional parenthesis) of the      %
% theorem.                                                                     %
%%%%%%%%%%%%%%%%%%%%%%%%%%%%%%%%%%%%%%%%%%%%%%%%%%%%%%%%%%%%%%%%%%%%%%%%%%%%%%%%
 
\newtheoremstyle{thm}       % name
    {15pt}                  % Space above
    {10pt}                  % Space below
    {\itshape}              % Body font
    {}                      % indent amount
    {\bf}                   % Theorem head font
    {}                      % Punctuation after theorem head
    {\newline}              % space after theorem head
    {}                      % Theorem head spec

\newtheoremstyle{def}
    {15pt}
    {10pt}
    {}
    {}
    {\bf}
    {}
    {\newline}
    {}

%%%%%%%%%%%%%%%%%%%%%%%%%%%%%%%%%%%%%%%%%%%%%%%%%%%%%%%%%%%%%%%%%%%%%%%%%%%%%%%%
% The following commands provide enumeration features that yield a section-    %
% based numbering, which is continous through all defined environments. This   %
% is achieved by anchoring one environment to the section numbering. All       %
% subsequent environment are numbered in the same fashion as the single one.   %
%%%%%%%%%%%%%%%%%%%%%%%%%%%%%%%%%%%%%%%%%%%%%%%%%%%%%%%%%%%%%%%%%%%%%%%%%%%%%%%%

\swapnumbers

\theoremstyle{thm}

\newtheorem{intTheorem}{Satz}[subsection]             % Theorem
\newtheorem{intCorollary}[intTheorem]{Korollar}    % Corollary
\newtheorem{intLemma}[intTheorem]{Lemma}           % Lemma

\theoremstyle{def}
\newtheorem{intDefinition}[intTheorem]{Definition} % Definition

%%%%%%%%%%%%%%%%%%%%%%%%%%%%%%%%%%%%%%%%%%%%%%%%%%%%%%%%%%%%%%%%%%%%%%%%%%%%%%%%
% Actual environments. The parameters are:                                     %
% #1 - name of the block (e.g. term to be defined)                             %
% #2 - label of the block (purely for internal reference purposes)             %
%%%%%%%%%%%%%%%%%%%%%%%%%%%%%%%%%%%%%%%%%%%%%%%%%%%%%%%%%%%%%%%%%%%%%%%%%%%%%%%%

\newenvironment{theorem}[2] % 1=Name, 2=Label
  {\begin{intTheorem}[#1] 
   \label{#2}
   %\addcontentsline{toc}{section}{\protect\numberline{\ref{#2}} #1}
   } 
  {\end{intTheorem}}

\newenvironment{corollary}[2] % 1=Name, 2=Label
  {\begin{intCorollary}[#1]
   \label{#2}
   %\addcontentsline{toc}{section}{\protect\numberline{\ref{#2}} #1}
   }
  {\end{intCorollary}}

\newenvironment{lemma}[2] % 1=Name, 2=Label
  {\begin{intLemma}[#1]
   \label{#2}
   %\addcontentsline{toc}{section}{\protect\numberline{\ref{#2}} #1}
   } 
  {\end{intLemma}}
 
\newenvironment{definition}[2] % 1=Name, 2=Label
  {\begin{intDefinition}[#1]
   \label{#2}
   %\addcontentsline{toc}{section}{\protect\numberline{\ref{#2}} {#1}}
   }
  {\end{intDefinition}}